\documentclass{article}
% pre\'ambulo

\usepackage{lmodern}
\usepackage[T1]{fontenc}
\usepackage{graphicx}
\usepackage[spanish,activeacute]{babel}
\usepackage{mathtools}

\title{Quintom.}
\author{Jonathan Rinc\'on Saucedo}



\begin{document}
\maketitle
% cuerpo del documento

  
    \maketitle
    \begin{abstract}
    Este trabajo tiene la finalidad de presentar un candidato potencial para la energ\'ia oscura denominado Quintom. Haremos un breve repaso en la historia sobre como se tuvo que plantear la idea de una energ\'ia que aceleraba el Universo, as\'i como los modelos m\'as simples en la cosmolog\'ia que tratan de solucionar el problema de la expansi\'on. Nos concentraremos en los modelos Quintessence y Phantom los cuales nos daran las bases para la comprensi\'on de nuestro modelo Quintom, el cual resulta la mejor opci\'on de tratar de resolver el problema.
    \end{abstract}



\section*{Energ\'ia Oscura} 
Uno de los mayores retos para la cosmolog\'ia es determinar la composici\'on del Universo. Una de esas componentes quiz\'a la m\'as importante es la energ\'ia oscura.
Una extra\~na energ\'ia que impulsa al Universo a tener una expansi\'on acelerada.
El primier indicio fue en 1917, cuando Eintein agreg\'o un t\'ermino a sus ecuaciones que denomin\'o constante cosmol\'ogica, con el prop\'osito de mantener un Universo est\'atico. Lamentablemente Einstein admitir\'ia unos a\~nos despu\'es que hab\'ia sido el mayor error de su vida ya que se demostr\'o que el universo estaba en expansi\'on y no solo eso si no que lo hac\'ia aceleradamente.  


\section*{FRW}
Para poder modelar nuestro Universo hay que tener muchas consideraciones en cuenta, el modelo FRW (Friedmann-Robertson-Walker) considera un Universo homog\'eneo e is\'otropo. 
Donde por homogeneidad nos referimos a que todos los puntos son equivalentes, e isotrop\'ia se refiere a que todas las direcci\'ones son equivalentes en una hipersuperficie particular. 

Donde la m\'etrica es

\begin{equation}
ds^2= -dt^2 + a^{2}(t) [\frac{dr^2}{1-k r^2} +r^2(d \theta^{2}+sin^{2} \theta d\varphi^{2})] 
\end{equation}

Si introducimos la m\'etrica anterior en las ecuaciones de Einstein obtenemos las ecuaciones de Friedmann:

\begin{equation}
3M_{P} ^{2} H^{2}=\rho
\end{equation}

\begin{equation}
-2M_{P} ^{2} \dot{H}=\rho + p
\end{equation}

Donde $\dot{H}=\frac{\dot{a}}{a}$, si combinamos las ecuaciones anteriores obtenemos 

\begin{equation}
\frac{\ddot{a}}{a} = -\frac{\rho + 3p}{6 M_{P} ^{2}}
\end{equation}

cuando $p> - \frac{\rho}{3}$ el Universo desacelera mientras que para $p< -\frac{\rho}{3}$ acelera.
Adem\'as definimos la densidad de materia, radiaci\'on y energ\'ia oscura como:

\begin{equation}
\Omega_{m} = \frac{\rho_{m}}{\rho_{c}};\\ \Omega_{r}=\frac{\rho_{r}}{\rho_{c}};\\ \Omega_{de}=\frac{\rho_{de}}{\rho_{c}}
\end{equation}

 
\section*{Quintessece} 
En la literatura Quintessence es una componente negativa de la presi\'on del fluido cosmico con las caracteristicas siguientes:
var\'ia en el tiempo y es espacialmente inhomogeneo. Esta es su principal diferencia con el modelo de $\Lambda$ que es espacialmente dina\'amica.
En general el campo de quintessence es un campo escalar con un t\'ermino de energ\'ia cin\'etica m\'inimamente acoplado a la gravedad. La acci\'on de la parte escalar esta dada por:

\begin{equation}
S=\int d^{4} x \sqrt{-g} [-\frac{1}{2} g^{\mu \nu} \partial_{\mu}\phi \partial_{\nu}\phi - V(\phi)]
\end{equation}

Donde hemos puesto como convenci\'on la m\'etrica (-,+,+,+) de modo que el campo escalar tenga un t\'ermino cin\'etico est\'andar.
Si tomamos la variaci\'on de $g^{\mu \nu}$ podemos obtener la ecuaci\'on para el tensor de energ\'ia momento dada por:

\begin{equation}
T_{\mu \nu}=\partial_{\mu} \phi \partial_{\nu} \phi - g_{\mu \nu} [\frac{1}{2} \partial^{\lambda} \phi \partial_{\lambda} \phi + V(\phi) ]
\end{equation}

En donde la densidad de energ\'ia y presi\'on son:

\begin{equation}
\rho_{de} =\frac{1}{2} \dot{\phi^{2}} + V(\phi)
\end{equation} 

\begin{equation}
p_{de}=\frac{1}{2} \dot{\phi^{2}} -V(\phi)
\end{equation}

Y las ecuaci\'ones de Fredmann son:

\begin{equation}
3M_{P} ^{2} H^{2} =\frac{1}{2} \dot{\phi^{2}} + V(\phi)
\end{equation}

\begin{equation}
-2M_{P} ^{2} \dot{H} = \rho_{de} + p_{de}
\end{equation}

donde recordemos que $M_{P} ^{2}=\frac{1}{8\pi G}$. Sustituyendo las ecuaciones (3) y (4) obtenemos

\begin{equation}
3M_{P}^{2} H^{2} = \frac{1}{2} \dot{\phi^{2}} + V(\phi)       
\end{equation}

\begin{equation}
-2M_{P}^{2} \dot{H}=\dot{\phi^{2}}
\end{equation}

La ecuaci\'on de estado tiene la forma 

\begin{equation}
w_{de}=\frac{p_{de}}{\rho_{de}}=\frac{\frac{1}{2} \dot{\phi^{2}} -V(\phi)}{\frac{1}{2} \dot{\phi^{2}} + V(\phi)}
\end{equation}



\section*{Phantom}
Al igual que Quintessence el modelo Phantom es un modelo para la energ\'ia oscura. Su pecurialidad es que posee un termino de energ\'ia cin\'etica negativa. Como en el caso anterior la acci\'on esta determinada por

\begin{equation}
S=\int d^{4} x \sqrt{-g} [-\frac{1}{2} g^{\mu \nu} \partial_{\mu}\phi \partial_{\nu}\phi - V(\phi)]
\end{equation}

Pero en este caso la parte cin\'etica del lagrangiano es la forma $\mathcal{L}_{kinetic} \propto -\dot{\psi^{2}}$. 
Para phantom la densidad de energ\'ia y presi\'on son

\begin{equation}
\rho_{de} =- \frac{1}{2} \dot{\phi^{2}} + V(\phi)
\end{equation}

\begin{equation}
p_{de} =- \frac{1}{2} \dot{\phi^{2}} - V(\phi)
\end{equation}

En donde obtenemos la ecuacion de estado

\begin{equation}
w_{de}=\frac{p_{de}}{\rho_{de}}=\frac{\frac{1}{2} \dot{\phi^{2}} + V(\phi)}{\frac{1}{2} \dot{\phi^{2}} - V(\phi)}
\end{equation}

En donde obtenemos que hay dos posibles resultados para la ecuaci\'on de estado: la primera es que $w_{de} > 1$ en donde la parte cin\'etica domina y el segundo caso cuando $w_{de} < -1$ en donde la parte del potencial domina.

Este \'ultimo caso tiene un comportamiento muy peculiar ya que act\'ua como una componente de la energ\'ia oscura con super aceleraci\'on. Es decir el Universo tendr\'ia una aceleraci\'on mucho m\'as r\'apida que una exponencial. Cuando ocurre esto la densidad de energ\'ia crece hasta que alcanza el infinito de manera que la tasa de expansi\'on diverge y se produce un fen\'omeno denominado "Big Rip".


\section{Quintom}
Sabemos que el campo Quintessence siempre tiene $w_{de} > -1$ y el campo Phantom $w_{de} < -1$. Si hacemos una combinaci\'on de estos dos modelos anteriores para obtener una $w_{de}=-1$, obtenemos lo que se denomina energ\'ia oscura Quintom.

Un detalle importante es que no es posible s\'olo con un campo escalar  cruzar la linea divisora phantom, por lo que nos vemos obligados a considerar modelos con al menos dos campos escalares, con la acci\'on

\begin{equation}
S=\int d^{4} x \sqrt{-g} [-\frac{1}{2} \partial^{\mu}\phi \partial_{\mu}\phi + \frac{1}{2} \partial ^{\mu} \sigma \partial_{\mu} \sigma - V(\phi,\sigma)]
\end{equation}

Donde la ecuaci\'on de estado vendr\a dada por

\begin{equation}
w_{de}=\frac{\frac{1}{2} \dot{\phi^{2}} - \frac{1}{2} \dot{\sigma^{2}} - V(\phi,\sigma)} {\frac{1}{2} \dot{\phi^{2}} - \frac{1}{2} \dot{\sigma^{2}} + V(\phi,\sigma)}
\end{equation}


\end{document}
